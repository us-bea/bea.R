\nonstopmode{}
\documentclass[a4paper]{book}
\usepackage[times,inconsolata,hyper]{Rd}
\usepackage{makeidx}
\makeatletter\@ifl@t@r\fmtversion{2018/04/01}{}{\usepackage[utf8]{inputenc}}\makeatother
% \usepackage{graphicx} % @USE GRAPHICX@
\makeindex{}
\begin{document}
\chapter*{}
\begin{center}
{\textbf{\huge Package `bea.R'}}
\par\bigskip{\large \today}
\end{center}
\ifthenelse{\boolean{Rd@use@hyper}}{\hypersetup{pdftitle = {bea.R: Bureau of Economic Analysis API}}}{}
\ifthenelse{\boolean{Rd@use@hyper}}{\hypersetup{pdfauthor = {Andrea Batch}}}{}
\begin{description}
\raggedright{}
\item[Title]\AsIs{Bureau of Economic Analysis API}
\item[Version]\AsIs{1.1.0}
\item[Author]\AsIs{Andrea Batch [aut, cre], 
Jeff Chen [ctb],
Walt Kampas [ctb]}
\item[Maintainer]\AsIs{Andrea Batch }\email{andrea.batch@bea.gov}\AsIs{}
\item[Depends]\AsIs{R (>= 3.2.1),
data.table}
\item[Imports]\AsIs{httr,
DT,
shiny,
jsonlite,
googleVis,
shinydashboard,
ggplot2,
stringr,
chron,
gtable,
scales,
htmltools,
httpuv,
xtable,
stringi,
magrittr,
htmlwidgets,
Rcpp,
munsell,
colorspace,
plyr,
yaml}
\item[Description]\AsIs{Provides an R interface for the Bureau of Economic Analysis (BEA) 
            API (see <}\url{http://www.bea.gov/API/bea_web_service_api_user_guide.htm}\AsIs{> for 
            more information) that serves two core purposes - 
1. To Extract/Transform/Load data [beaGet()] from the BEA API as R-friendly 
            formats in the user's work space [transformation done by default in beaGet() 
            can be modified using optional parameters; see, too, bea2List(), bea2Tab()].
            2. To enable the search of descriptive meta data [beaSearch()].
            Other features of the library exist mainly as intermediate methods 
            or are in early stages of development.
            Important Note - You must have an API key to use this library.  
            Register for a key at <}\url{http://www.bea.gov/API/signup/index.cfm}\AsIs{> .}
\item[URL]\AsIs{}\url{https://github.com/us-bea/bea.R}\AsIs{}
\item[License]\AsIs{CC0}
\item[LazyData]\AsIs{no}
\item[Encoding]\AsIs{UTF-8}
\item[Roxygen]\AsIs{list(markdown = TRUE)}
\item[RoxygenNote]\AsIs{7.3.2}
\end{description}
\Rdcontents{Contents}
